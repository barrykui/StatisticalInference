% !TEX encoding = UTF-8 Unicode
% !TEX TS-program = XeLaTeX
\documentclass [12pt]{article}
\usepackage[Lenny]{fncychap}
\usepackage{mathrsfs}
\usepackage{amsmath}
\usepackage{amssymb}
\usepackage{graphics}
\textwidth=6.5in \textheight=9in \topmargin=1pt \oddsidemargin=0pt
\evensidemargin=0pt
\renewcommand{\baselinestretch}{1.5}
\usepackage{CJK}
\usepackage{fontspec,xltxtra,xunicode}  
\defaultfontfeatures{Mapping=tex-text}  
%\setromanfont{SimSun} %设置中文字体  
\setromanfont{Kai} %设置中文字体  
\XeTeXlinebreaklocale “zh”  
\XeTeXlinebreakskip = 0pt plus 1pt minus 0.1pt %文章内中文自动换行  
\usepackage{multirow}


%-----------------------------------------------




%---------------------------------------------------
\def\hilite<#1>{\temporal<#1>{\color{blue!35}}{\color{magenta}}{\color{blue!75}}}
%% 自定义命令, 源自 beamer_guide. item 逐步显示时, 使已经出现的item、正在显示的item、将要出现的item 呈现不同颜色.
%%%%%%%%%%%%%%%%%%%%%%%%%%%%%%%%%%%%%%%%%%%%%%%%%%%%%%%%%%%%%%%%%%%%%%%%%%%%%%%%%%%%%%%%%%%%%%%%%%%%%%%%
\def\la{\lambda}
\def\bfta{{\bm{\theta}}}
\def\Ta{{\Theta}}
\def\ta{{\theta}}
\def\hta{{\hat{\theta}}}
\def\y1n{{Y_1,\cdots, Y_n}}
\def\x1n{{X_1,\cdots, X_n}}
\def\f12{{\frac{1}{2}}}
\def\barx{{\bar{X}}}
\def\f12{{\frac{1}{2}}}
\def\Sig{{\Sigma}}
\def\sig{{\sigma}}
\def\R{{\mathcal{R}}}
\def\p{{\mathcal{P}}}
\def\f{{\mathcal{F}}}
\def\e{{\epsilon}}
\def\barx{{\bar{X}}}
\def\bx{{\bm{x}}}
\def\bX{{\bm{\mathcal{X}}}}

\def\v#1{\stackrel{#1}{\longrightarrow}}
%---------------------------------------------
\newcommand{\cov}{\mbox{cov}}
\newcommand{\Var}{\mbox{Var}}
\newcommand{\E}{\mbox{E}}
%----------------------------------------------------
\newcommand{\limn}{{\lim\limits_{n\to\infty}}}
\newcommand{\liminfn}{{\liminf\limits_{n\to\infty}}}
\newcommand{\limsupn}{{\limsup\limits_{n\to\infty}}}

\newcommand{\sumx}[1]{\sum\limits_{i=1}^n {#1}_i}
\newcommand{\sumn}[1]{\sum\limits_{#1=1}^n}
%----------------------------------------------------
\newcommand{\timx}[1]{\prod\limits_{i=1}^n {#1}_i}
\newcommand{\timn}[1]{\prod\limits_{#1=1}^n}
%-------------------------------------------------------
\def\dps{\displaystyle}
\def\Om{{\Omega}}
\def\om{{\omega}}
\def\pp{{\mathcal{P}}}
\def\f{{\mathcal{F}}}
\def\R{{\mathcal{R}}}
\def\B{{\mathcal{B}}}
\def\A{{\mathcal{A}}}
\def\X{{\mathcal{X}}}
\def\C{{\mathcal{C}}}
\def\D{{\mathcal{D}}}
\def\O{{\mathcal{O}}}
\def\J{{\mathcal{J}}}
\def\Mu{{\mathcal{U}}}
%------------------------------------------
\def\b{{\beta}}
\def\a{{\alpha}}
\def\e{{\epsilon}}
\def\la{\lambda}
\def\La{\Lambda}
\def\ta{{\theta}}
\def\Ta{{\Theta}}
%-------------------------------------

\def\Sig{\Sigma}
\def\sig{\sigma}

\def\dfrac#1#2{{\displaystyle{#1\over#2}}}
\def\indA#1#2{{A^*_{#1 #2}(t)=a_m}}
\def\sumL {{\sum\limits_{m=1}^L}}
\def\dbN#1{{d \bar{N}(#1)}}


\def\Y#1#2{{Y_{#1#2}(t)}}
\def\Iam#1#2{{I(A^*_{#1#2}=a_m)}}


%------------------------------------------
\newcommand{\xn}[1]{ #1_1, \cdots, #1_n}
\newcommand{\capn}[1]{\cap\limits_{#1=1}^n}
\newcommand{\enorm}[2]{(E({#1}^#2))^{1/#2}}

%\newcommand{\sumx}[1]{\sum\limits_{i=1}^n {#1}_i}
\newcommand{\seqn}[1]{#1_1,#1_2,\cdots}

%%%%%%%%%%%%%%%%%%%%%%%%%%%%
\newtheorem{exam}{example}[section]
\newtheorem{df}{definition}[section]
\newtheorem{thm}{Theorem}[section]
\newtheorem{lem}{Lemma}[section]
\newtheorem{cor}{Corrolary}[section]
\newtheorem{rem}{Remark}[section]
%------------------------------------------------------------------
\begin{document}
 \begin{CJK*}{GBK}{kai}
 %%----------------------- Theorems ---------------------------------------------------------------------
\newtheorem{theorem}{定理}
\newtheorem{definition}{定义}
%\newtheorem{df}{定义}
\newtheorem{lemma}{引理}
\newtheorem{corollary}{推论}
\newtheorem{proposition}{性质}
\newtheorem{example}{例}
\newtheorem{remark}{注}
%%----------------------------------------------------------------------------------------------------
    \title{2016年秋《统计方法与应用》作业-4(Random Sampling)}
    \author{ 姓名:徐魁\,\,\,\, 学号~~{2016311209}}
    \date{\today}
\maketitle

%%---------------------------------------------------------------------------------------------------
\section{Reading. }
%===================================================================================================
\begin{enumerate}
  \item[(a)] Lecture notes 4.
 \item[(b)] Chpaters 5 of the book ”Statistical Inference”.\\
 
\end{enumerate}

%%---------------------------------------------------------------------------------------------------
\section{Let $X_1, \ldots, X_n$ be a random sample of size $n$ from a $N(0, \sigma^2)$ population. Prove that
        \[
            \frac{(n-1)S^2}{\sigma^2} = \sum_{i=1}^n\Big(\frac{X_i-\bar{X}}{\sigma}\Big)^2
        \]
        has a $\chi^2$ distribution with $n-1$ degrees of freedom.}%===================================================================================================
	证明:
		现在要证明$\frac{(n-1)S^2}{\sigma^2} = \sum_{i=1}^n\Big(\frac{X_i-\bar{X}}{\sigma}\Big)^2$有一个 $\chi^2$ 分布和 $n-1$ 的自由度,我们采用数学归纳法证明,有递推公式可知,
		\[
            		(n-1)S^2 = (n-2)S_{n-1}^2+\Big(\frac{n-1}{n}\Big)(X_{n}- \bar{X}_{n-1})^2
        		\]
		当n=2时,有,
		\[
			S^2 = \frac{1}{2}(X_2-X_1)^2
		\]
		由于$(X_2-X_1)/\sqrt{2}$服从于n(0,1)分布,于是根据引理5.3.2有$S_2^2 \thicksim \chi_1^2$。\\
		现假设$(k-1)S_{k}^2 \thicksim \chi_{k-1}^2$,\\
		则,当对于$n=k+1$,有,
		\[
            		kS^2 = (k-2)S_{k}^2+\Big(\frac{k}{k+1}\Big)(X_{k+1}- \bar{X}_{k})^2
        		\]
		根据,假设$(k-1)S_{k}^2 \thicksim \chi_{k-1}^2$,且根据引理5.3.2,$kS_{k}^2 \thicksim \chi_{k}^2$,从而定理得证。\\
		另外,可由定理:$X_1,...,X_n$是一列独立的随机向量,$g_i(x_i)$是$x_i$的一元函数,则随机变量$U_i=g_i(X_i)$直接证明,$(X_{k+1}- \bar{X}_{k})^2  \thicksim S_{k}^2 $的独立性。
		
		


  
%%---------------------------------------------------------------------------------------------------
\section{Let $X_1, \ldots, X_n$ be a random sample of size $n$ from a $N(0, \sigma^2)$ population. Prove that $\bar{X}$ and $S^2$ are independent random variables. }
%===================================================================================================


%%%%%%%%%%%%%%%%%%%%%%%%%%%%%%%%%%%%%%%%%%%%%%%%%%%%%%%%%%%%%%%%%%%%%%%%%%%%%%%%%%%%%%%%%%%%%%
  \end{CJK*}
\end{document}
