% !TEX encoding = UTF-8 Unicode
% !TEX TS-program = XeLaTeX
\documentclass [12pt]{article}
\usepackage{mathrsfs}
\usepackage{amsmath}
\usepackage{amssymb}
\usepackage{graphics}
\textwidth=6.5in \textheight=9in \topmargin=1pt \oddsidemargin=0pt
\evensidemargin=0pt
\renewcommand{\baselinestretch}{1.5}
\usepackage{CJK}
\usepackage{fontspec,xltxtra,xunicode}  
\defaultfontfeatures{Mapping=tex-text}  
%\setromanfont{SimSun} %设置中文字体  
\setromanfont{Kai} %设置中文字体  
\XeTeXlinebreaklocale “zh”  
\XeTeXlinebreakskip = 0pt plus 1pt minus 0.1pt %文章内中文自动换行  
  
%-----------------------------------------------




%---------------------------------------------------
\def\hilite<#1>{\temporal<#1>{\color{blue!35}}{\color{magenta}}{\color{blue!75}}}
%% 自定义命令, 源自 beamer_guide. item 逐步显示时, 使已经出现的item、正在显示的item、将要出现的item 呈现不同颜色.
%%%%%%%%%%%%%%%%%%%%%%%%%%%%%%%%%%%%%%%%%%%%%%%%%%%%%%%%%%%%%%%%%%%%%%%%%%%%%%%%%%%%%%%%%%%%%%%%%%%%%%%%
\def\la{\lambda}
\def\bfta{{\bm{\theta}}}
\def\Ta{{\Theta}}
\def\ta{{\theta}}
\def\hta{{\hat{\theta}}}
\def\y1n{{Y_1,\cdots, Y_n}}
\def\x1n{{X_1,\cdots, X_n}}
\def\f12{{\frac{1}{2}}}
\def\barx{{\bar{X}}}
\def\f12{{\frac{1}{2}}}
\def\Sig{{\Sigma}}
\def\sig{{\sigma}}
\def\R{{\mathcal{R}}}
\def\p{{\mathcal{P}}}
\def\f{{\mathcal{F}}}
\def\e{{\epsilon}}
\def\barx{{\bar{X}}}
\def\bx{{\bm{x}}}
\def\bX{{\bm{\mathcal{X}}}}

\def\v#1{\stackrel{#1}{\longrightarrow}}
%---------------------------------------------
\newcommand{\cov}{\mbox{cov}}
\newcommand{\Var}{\mbox{Var}}
\newcommand{\E}{\mbox{E}}
%----------------------------------------------------
\newcommand{\limn}{{\lim\limits_{n\to\infty}}}
\newcommand{\liminfn}{{\liminf\limits_{n\to\infty}}}
\newcommand{\limsupn}{{\limsup\limits_{n\to\infty}}}

\newcommand{\sumx}[1]{\sum\limits_{i=1}^n {#1}_i}
\newcommand{\sumn}[1]{\sum\limits_{#1=1}^n}
%----------------------------------------------------
\newcommand{\timx}[1]{\prod\limits_{i=1}^n {#1}_i}
\newcommand{\timn}[1]{\prod\limits_{#1=1}^n}
%-------------------------------------------------------
\def\dps{\displaystyle}
\def\Om{{\Omega}}
\def\om{{\omega}}
\def\pp{{\mathcal{P}}}
\def\f{{\mathcal{F}}}
\def\R{{\mathcal{R}}}
\def\B{{\mathcal{B}}}
\def\A{{\mathcal{A}}}
\def\X{{\mathcal{X}}}
\def\C{{\mathcal{C}}}
\def\D{{\mathcal{D}}}
\def\O{{\mathcal{O}}}
\def\J{{\mathcal{J}}}
\def\Mu{{\mathcal{U}}}
%------------------------------------------
\def\b{{\beta}}
\def\a{{\alpha}}
\def\e{{\epsilon}}
\def\la{\lambda}
\def\La{\Lambda}
\def\ta{{\theta}}
\def\Ta{{\Theta}}
%-------------------------------------

\def\Sig{\Sigma}
\def\sig{\sigma}

\def\dfrac#1#2{{\displaystyle{#1\over#2}}}
\def\indA#1#2{{A^*_{#1 #2}(t)=a_m}}
\def\sumL {{\sum\limits_{m=1}^L}}
\def\dbN#1{{d \bar{N}(#1)}}


\def\Y#1#2{{Y_{#1#2}(t)}}
\def\Iam#1#2{{I(A^*_{#1#2}=a_m)}}


%------------------------------------------
\newcommand{\xn}[1]{ #1_1, \cdots, #1_n}
\newcommand{\capn}[1]{\cap\limits_{#1=1}^n}
\newcommand{\enorm}[2]{(E({#1}^#2))^{1/#2}}

%\newcommand{\sumx}[1]{\sum\limits_{i=1}^n {#1}_i}
\newcommand{\seqn}[1]{#1_1,#1_2,\cdots}

%%%%%%%%%%%%%%%%%%%%%%%%%%%%
\newtheorem{exam}{example}[section]
\newtheorem{df}{definition}[section]
\newtheorem{thm}{Theorem}[section]
\newtheorem{lem}{Lemma}[section]
\newtheorem{cor}{Corrolary}[section]
\newtheorem{rem}{Remark}[section]
%------------------------------------------------------------------
\begin{document}
 \begin{CJK*}{GBK}{kai}
 %%----------------------- Theorems ---------------------------------------------------------------------
\newtheorem{theorem}{定理}
\newtheorem{definition}{定义}
%\newtheorem{df}{定义}
\newtheorem{lemma}{引理}
\newtheorem{corollary}{推论}
\newtheorem{proposition}{性质}
\newtheorem{example}{例}
\newtheorem{remark}{注}
%%----------------------------------------------------------------------------------------------------
    \title{2016年秋《统计方法与应用》作业-1(概率论)}
    \author{ 姓名:徐魁\,\,\,\, 学号~~{2016311209}}
    \date{\today}
\maketitle

%%---------------------------------------------------------------------------------------------------
\section{证明下列恒等式(1.2) }
%===================================================================================================
\begin{enumerate}
  \item[(a)] $A \setminus B = A \setminus (A \cap B) = A \cap B^{C} $\\
 证明:分两步进行证明,首先证明$A \setminus B = A \setminus (A \cap B) $. \\
 令$x \in A \setminus B$, 即$x \in A$ 且 $x \notin B$. \\
 由$x \notin B$ 就有$x \notin ( A \cap B)$,\\
 进而就有 $x \in A$ 且 $x \notin ( A \cap B)$,\\
 因此$A \setminus B = A \setminus (A \cap B) $。
 故"$A \setminus B = A \setminus (A \cap B) $"得证。\\
 \\
 同样,由$x \notin B$ 就有$x \in B^{C}$,\\
进而就有 $x \in A$ 且 $x \in B^{C}$,\\
 因此$A \setminus B = A \cap B^{C} $。\\
 故“$A \setminus B = A \cap B^{C}  $”得证。\\
 \\
 从而,$A \setminus B = A \setminus (A \cap B) = A \cap B^{C} $得证。
 
 \item[(b)] $B=(B \cap A) \cup (B \cap A^{C})$\\
 证明:两种方法均可证明。\\
 方法一:课本定理1.1.4直接推导。\\
由$\textbf{定理 1.1.4}$ 分配律可得,\\
$(B \cap A) \cup (B \cap A^{C}) $ \\
$\Leftrightarrow$\\
 $((B \cap A) \cup B) \cap ( (B \cap A) \cup A^{C})$ (分配律)\\
 $\Leftrightarrow$\\
 $B \cap ( (B \cap A) \cup A^{C})$\\
 $\Leftrightarrow$\\
 $B \cap ( (B \cup A^{C}) \cap (A \cup A^{C}))$ (结合律)\\
 $\Leftrightarrow$\\
 $B \cap (B \cup A^{C})$ \\
 $\Leftrightarrow$\\
$B$
故“$B=(B \cap A) \cup (B \cap A^{C})$”得证。\\
\\
 方法二:用集合包含关系证明。\\
 分别证明两个集合的互相之间的包含关系。即要证明:$B \subset (B \cap A) \cup (B \cap A^{C})$ 和 $ (B \cap A) \cup (B \cap A^{C}) \subset B$\\
  先证明$B \subset (B \cap A) \cup (B \cap A^{C})$ 。\\
  令$x \in B$, 有$x \in A$或$x \in A^{C}$。若$x \in A$,则$x \in B$且$x \in A$,进而$x \in (B \cap A) $。若$x \in A^{C}$,则$x \in B$且$x \in A^{C}$,进而$x \in (B \cap A^{C}) $。因此有$x \in  (B \cap A) \cup (B \cap A^{C})$, 故$B \subset (B \cap A) \cup (B \cap A^{C})$得证。\\
  再证明$ (B \cap A) \cup (B \cap A^{C}) \subset B$。\\
  令$x \in (B \cap A) \cup (B \cap A^{C})$。由$x \in (B \cap A) $,而$(B \cap A) \subset B $,故$x \in B$。由$x \in (B \cap A^{C}) $,而$(B \cap A^{C}) \subset B $,故$x \in B$。因此$ (B \cap A) \cup (B \cap A^{C}) \subset B$得证。\\
  故“$B=(B \cap A) \cup (B \cap A^{C})$”得证。\\
 % ===================== C ================
   \item[(c)] $B \setminus A = B \cap A^{C}$\\
 证明:由$x \notin A$ 就有$x \in A^{C}$,\\
进而就有 $x \in B$ 且 $x \in A^{C}$,\\
 因此$B \setminus A = B \cap A^{C}$。\\
 故“$B \setminus A = B \cap A^{C}$”得证。\\

 % ===================== D ================
   \item[(c)] $A \cup B = A \cup (B \cap A^{C})$\\
   证明:课本定理1.1.4直接推导。\\
   $ A \cup (B \cap A^{C})$\\
   $\Leftrightarrow$\\
   $(A \cup B) \cap (A \cup A^{C}))$ (结合律)\\
   $\Leftrightarrow$\\
   $(A \cup B) \cap S$\\
   $\Leftrightarrow$\\
   $A \cup B $
\end{enumerate}

%===================================================================================================
\begin{enumerate}
  \item[a] 假设样本~$\x1n$ 来自指数总体 $e(\lambda)$,
  试验进行到~$r$~个产品失效为止,写出似然函数,并估计未知参数~$\lambda$.
  \item   假设样本~$\x1n$ 来自指数总体 $e(\lambda)$,
  试验进行~$t$~ 时刻为止,写出似然函数,并估计未知参数~$\lambda$.
  \item 解:指数分布的密度函数:
$$f(x;\lambda)=\lambda \exp\{-\lambda x\} I(x>0).$$
似然函数:
\begin{eqnarray*}
% \nonumber to remove numbering (before each equation)
  L(\lambda)&=&\timn i \left({\lambda \exp\{-\lambda X_i\}}\right)^{d_i} (\exp\{-\lambda t\})^{1-d_i} \\
   &=& \lambda^{ \sumn i d_i} \exp\{-\lambda \sumn i [d_i X_i+(1-d_i)t]\}\\
  l(\la)&=&\log L(\la)=\sumn i d_i \cdot \log \la -\la \sumn i [d_i X_i+(1-d_i)t]
\end{eqnarray*}

\end{enumerate}


  %------------------------------------------------------

$$ \hat{\la}=\frac{\sumn i d_i}{\sumn i [d_i X_i+(1-d_i)t]}.$$

问题:(1)$\hat{\la}$~的渐进分布?或者渐进方差=?如何估计渐进方差?\\
(2) 写一个子程序估计上面对渐进方差 $\hat{\sig}$.\\
(3) 在模拟计算中, 用 $S$~向量记录每次模拟的方差估计,
之后用~$mean(S)$~算500次模拟的方差的平均值,与参数估计的真实样本方差
($var(lamda)$, 其中 lamda 用来记录500次模拟的参数$\la$~的估计。



%%%%%%%%%%%%%%%%%%%%%%%%%%%%%%%%%%%%%%%%%%%%%%%%%%%%%%%%%%%%%%%%%%%%%%%%%%%%%%%%%%%%%%%%%%%%%%
  \end{CJK*}
\end{document}
