% !TEX encoding = UTF-8 Unicode
% !TEX TS-program = XeLaTeX
\documentclass [12pt]{article}
\usepackage{mathrsfs}
\usepackage{amsmath}
\usepackage{amssymb}
\usepackage{graphics}
\textwidth=6.5in \textheight=9in \topmargin=1pt \oddsidemargin=0pt
\evensidemargin=0pt
\renewcommand{\baselinestretch}{1.5}
\usepackage{CJK}
\usepackage{fontspec,xltxtra,xunicode}  
\defaultfontfeatures{Mapping=tex-text}  
%\setromanfont{SimSun} %设置中文字体  
\setromanfont{Kai} %设置中文字体  
\XeTeXlinebreaklocale “zh”  
\XeTeXlinebreakskip = 0pt plus 1pt minus 0.1pt %文章内中文自动换行  
\usepackage{multirow}


%-----------------------------------------------




%---------------------------------------------------
\def\hilite<#1>{\temporal<#1>{\color{blue!35}}{\color{magenta}}{\color{blue!75}}}
%% 自定义命令, 源自 beamer_guide. item 逐步显示时, 使已经出现的item、正在显示的item、将要出现的item 呈现不同颜色.
%%%%%%%%%%%%%%%%%%%%%%%%%%%%%%%%%%%%%%%%%%%%%%%%%%%%%%%%%%%%%%%%%%%%%%%%%%%%%%%%%%%%%%%%%%%%%%%%%%%%%%%%
\def\la{\lambda}
\def\bfta{{\bm{\theta}}}
\def\Ta{{\Theta}}
\def\ta{{\theta}}
\def\hta{{\hat{\theta}}}
\def\y1n{{Y_1,\cdots, Y_n}}
\def\x1n{{X_1,\cdots, X_n}}
\def\f12{{\frac{1}{2}}}
\def\barx{{\bar{X}}}
\def\f12{{\frac{1}{2}}}
\def\Sig{{\Sigma}}
\def\sig{{\sigma}}
\def\R{{\mathcal{R}}}
\def\p{{\mathcal{P}}}
\def\f{{\mathcal{F}}}
\def\e{{\epsilon}}
\def\barx{{\bar{X}}}
\def\bx{{\bm{x}}}
\def\bX{{\bm{\mathcal{X}}}}

\def\v#1{\stackrel{#1}{\longrightarrow}}
%---------------------------------------------
\newcommand{\cov}{\mbox{cov}}
\newcommand{\Var}{\mbox{Var}}
\newcommand{\E}{\mbox{E}}
%----------------------------------------------------
\newcommand{\limn}{{\lim\limits_{n\to\infty}}}
\newcommand{\liminfn}{{\liminf\limits_{n\to\infty}}}
\newcommand{\limsupn}{{\limsup\limits_{n\to\infty}}}

\newcommand{\sumx}[1]{\sum\limits_{i=1}^n {#1}_i}
\newcommand{\sumn}[1]{\sum\limits_{#1=1}^n}
%----------------------------------------------------
\newcommand{\timx}[1]{\prod\limits_{i=1}^n {#1}_i}
\newcommand{\timn}[1]{\prod\limits_{#1=1}^n}
%-------------------------------------------------------
\def\dps{\displaystyle}
\def\Om{{\Omega}}
\def\om{{\omega}}
\def\pp{{\mathcal{P}}}
\def\f{{\mathcal{F}}}
\def\R{{\mathcal{R}}}
\def\B{{\mathcal{B}}}
\def\A{{\mathcal{A}}}
\def\X{{\mathcal{X}}}
\def\C{{\mathcal{C}}}
\def\D{{\mathcal{D}}}
\def\O{{\mathcal{O}}}
\def\J{{\mathcal{J}}}
\def\Mu{{\mathcal{U}}}
%------------------------------------------
\def\b{{\beta}}
\def\a{{\alpha}}
\def\e{{\epsilon}}
\def\la{\lambda}
\def\La{\Lambda}
\def\ta{{\theta}}
\def\Ta{{\Theta}}
%-------------------------------------

\def\Sig{\Sigma}
\def\sig{\sigma}

\def\dfrac#1#2{{\displaystyle{#1\over#2}}}
\def\indA#1#2{{A^*_{#1 #2}(t)=a_m}}
\def\sumL {{\sum\limits_{m=1}^L}}
\def\dbN#1{{d \bar{N}(#1)}}


\def\Y#1#2{{Y_{#1#2}(t)}}
\def\Iam#1#2{{I(A^*_{#1#2}=a_m)}}


%------------------------------------------
\newcommand{\xn}[1]{ #1_1, \cdots, #1_n}
\newcommand{\capn}[1]{\cap\limits_{#1=1}^n}
\newcommand{\enorm}[2]{(E({#1}^#2))^{1/#2}}

%\newcommand{\sumx}[1]{\sum\limits_{i=1}^n {#1}_i}
\newcommand{\seqn}[1]{#1_1,#1_2,\cdots}

%%%%%%%%%%%%%%%%%%%%%%%%%%%%
\newtheorem{exam}{example}[section]
\newtheorem{df}{definition}[section]
\newtheorem{thm}{Theorem}[section]
\newtheorem{lem}{Lemma}[section]
\newtheorem{cor}{Corrolary}[section]
\newtheorem{rem}{Remark}[section]
%------------------------------------------------------------------
\begin{document}
 \begin{CJK*}{GBK}{kai}
 %%----------------------- Theorems ---------------------------------------------------------------------
\newtheorem{theorem}{定理}
\newtheorem{definition}{定义}
%\newtheorem{df}{定义}
\newtheorem{lemma}{引理}
\newtheorem{corollary}{推论}
\newtheorem{proposition}{性质}
\newtheorem{example}{例}
\newtheorem{remark}{注}
%%----------------------------------------------------------------------------------------------------
    \title{2016年秋《统计方法与应用》作业-1(概率论)}
    \author{ 姓名:徐魁\,\,\,\, 学号~~{2016311209}}
    \date{\today}
\maketitle

%%---------------------------------------------------------------------------------------------------
\section{【1.2】证明下列恒等式 }
%===================================================================================================
\begin{enumerate}
  \item[(a)] $A \setminus B = A \setminus (A \cap B) = A \cap B^{C} $\\
 证明:分两步进行证明,首先证明$A \setminus B = A \setminus (A \cap B) $. \\
 令$x \in A \setminus B$, 即$x \in A$ 且 $x \notin B$. \\
 由$x \notin B$ 就有$x \notin ( A \cap B)$,\\
 进而就有 $x \in A$ 且 $x \notin ( A \cap B)$,\\
 因此$A \setminus B = A \setminus (A \cap B) $。
 故"$A \setminus B = A \setminus (A \cap B) $"得证。\\
 \\
 同样,由$x \notin B$ 就有$x \in B^{C}$,\\
进而就有 $x \in A$ 且 $x \in B^{C}$,\\
 因此$A \setminus B = A \cap B^{C} $。\\
 故“$A \setminus B = A \cap B^{C}  $”得证。\\
 \\
 从而,$A \setminus B = A \setminus (A \cap B) = A \cap B^{C} $得证。
 
 \item[(b)] $B=(B \cap A) \cup (B \cap A^{C})$\\
 证明:两种方法均可证明。\\
 方法一:课本定理1.1.4直接推导。\\
由$\textbf{定理 1.1.4}$ 分配律可得,\\
$(B \cap A) \cup (B \cap A^{C}) $ \\
$\Leftrightarrow$\\
 $((B \cap A) \cup B) \cap ( (B \cap A) \cup A^{C})$ (分配律)\\
 $\Leftrightarrow$\\
 $B \cap ( (B \cap A) \cup A^{C})$\\
 $\Leftrightarrow$\\
 $B \cap ( (B \cup A^{C}) \cap (A \cup A^{C}))$ (结合律)\\
 $\Leftrightarrow$\\
 $B \cap (B \cup A^{C})$ \\
 $\Leftrightarrow$\\
$B$
故“$B=(B \cap A) \cup (B \cap A^{C})$”得证。\\
\\
 方法二:用集合包含关系证明。\\
 分别证明两个集合的互相之间的包含关系。即要证明:$B \subset (B \cap A) \cup (B \cap A^{C})$ 和 $ (B \cap A) \cup (B \cap A^{C}) \subset B$\\
  先证明$B \subset (B \cap A) \cup (B \cap A^{C})$ 。\\
  令$x \in B$, 有$x \in A$或$x \in A^{C}$。若$x \in A$,则$x \in B$且$x \in A$,进而$x \in (B \cap A) $。若$x \in A^{C}$,则$x \in B$且$x \in A^{C}$,进而$x \in (B \cap A^{C}) $。因此有$x \in  (B \cap A) \cup (B \cap A^{C})$, 故$B \subset (B \cap A) \cup (B \cap A^{C})$得证。\\
  再证明$ (B \cap A) \cup (B \cap A^{C}) \subset B$。\\
  令$x \in (B \cap A) \cup (B \cap A^{C})$。由$x \in (B \cap A) $,而$(B \cap A) \subset B $,故$x \in B$。由$x \in (B \cap A^{C}) $,而$(B \cap A^{C}) \subset B $,故$x \in B$。因此$ (B \cap A) \cup (B \cap A^{C}) \subset B$得证。\\
  故“$B=(B \cap A) \cup (B \cap A^{C})$”得证。\\
 % ===================== C ================
   \item[(c)] $B \setminus A = B \cap A^{C}$\\
 证明:由$x \notin A$ 就有$x \in A^{C}$,\\
进而就有 $x \in B$ 且 $x \in A^{C}$,\\
 因此$B \setminus A = B \cap A^{C}$。\\
 故“$B \setminus A = B \cap A^{C}$”得证。\\

 % ===================== D ================
   \item[(d)] $A \cup B = A \cup (B \cap A^{C})$\\
   证明:课本定理1.1.4直接推导。\\
   $ A \cup (B \cap A^{C})$\\
   $\Leftrightarrow$\\
   $(A \cup B) \cap (A \cup A^{C}))$ (结合律)\\
   $\Leftrightarrow$\\
   $(A \cup B) \cap S$\\
   $\Leftrightarrow$\\
   $A \cup B $
\end{enumerate}

%%---------------------------------------------------------------------------------------------------
\section{【1.4】设$A$,$B$为事件,利用$P(A)$,$P(B)$和$P(A \cap B)$,给出下列事件的概率公式:}
\begin{enumerate}
  \item[(a)]  事件为$A \cup B$,则概率为:$p=P(A \cup B) =  P(A) + P(B) - P(A \cap B) $(课本定理1.2.9-b)
  \item[(b)]  事件为$(A \cap B^{C}) \cup (A^{C} \cap B)$,则概率为:$p=P((A \cap B^{C}) \cup (A^{C} \cap B)) $\\
  = $P(A \cap B^{C}) + P(A^{C} \cap B) $(有限可加性公理)\\
  = $P(A) - P(A \cap B) + P(B) - P(A \cap B)  $(课本定理1.2.9-a)\\
  = $P(A) + P(B) - 2*P(A \cap B) $
  \item[(c)]  事件为$A \cup B$,则概率为:$p=P(A \cup B) =  P(A) + P(B) - P(A \cap B) $(课本定理1.2.9-b)
  \item[(d)]  事件$(A \cap B)^{C}$, 则概率为:$p=P((A \cap B)^{C}) =1 - P(A \cap B)  $(课本定理1.2.8-c)
  
\end{enumerate}
  
%%---------------------------------------------------------------------------------------------------
\section{【1.11】设S为样本空间, }
%===================================================================================================
\begin{enumerate}
  \item[(a)] 证明$\mathcal{B}=\{\varnothing,S\}$是一个$\sigma$代数;(证集合$\{\varnothing,S\}$满足$\sigma$代数的三个性质即可。)\\
  证明:\\对于集合$\{\varnothing,S\}$有$\varnothing \in \mathcal{B}$,即满足性质a。\\
  又因为$\varnothing \in \mathcal{B}$且$S \in \mathcal{B}$,而$S^{C}=\varnothing$,因此$\mathcal{B}$在补运算下封闭,即满足性质b。\\
  因为$(S \cup \varnothing) \in \mathcal{B}$,因此$\mathcal{B}$在并运算下封闭,即满足性质c。\\
  即证得$\mathcal{B}=\{\varnothing,S\}$是一个$\sigma$代数。
  \item[(b)] 令$\mathcal{B}=\{S的全体子集,包括S本身\}$,证明$\mathcal{B}$是一个$\sigma$代数;\\
  证明:\\因为$\varnothing$是任何集合的子集,那么$\varnothing \in S$, 因此$\varnothing \in \mathcal{B}$,即满足性质a。\\
  设$A$为$S$的一个子集,显然有$A \in \mathcal{B}$,而$A^{C} \in S$,因此$A^{C} \in \mathcal{B}$,即满足性质b。\\
  设$A_{1}, A_{2},...  $为$S$的所有的子集,有,$A_{1}, A_{2},...  \in S$。那么,显然有$A_{1}, A_{2},...  \in \mathcal{B}$, 即有$\bigcup\limits_{i=1}^{\infty}A_i \in \mathcal{B}$,即满足性质c。\\
   即证得$\mathcal{B}=\{S的全体子集,包括S本身\}$是一个$\sigma$代数。
  \item[(c)] 证明两个$\sigma$代数的交仍是$\sigma$代数。\\
  证明:\\设有两个$\sigma$代数$\mathcal{B}1$,$\mathcal{B}2$,由$\sigma$代数的性质a可知,$\varnothing \in \mathcal{B}1$且$\varnothing \in \mathcal{B}2$,进而可得$\varnothing \in \mathcal{B}1 \cap \mathcal{B}2$,即满足性质a。\\
  设$A \in (\mathcal{B}1 \cap \mathcal{B}2 )$, 则有$A \in \mathcal{B}1 $ 和$A \in \mathcal{B}2 $。由$\sigma$代数的性质b可知,$A^{C} \in \mathcal{B} 1$ 和$A^{C} \in \mathcal{B}2 $,因此$A^{C} \in \mathcal{B}1 \cap \mathcal{B}2 $,即满足性质b。\\
  设$A_{1}, A_{2},...  \in (\mathcal{B}1 \cap \mathcal{B}2 )$, 则有$A_{1}, A_{2},...  \in \mathcal{B}1 $ 和$A_{1}, A_{2},...  \in \mathcal{B}2 $。由$\sigma$代数的性质c可知,$\bigcup\limits_{i=1}^{\infty}A_i \in \mathcal{B}1 $ 和$\bigcup\limits_{i=1}^{\infty}A_i \in \mathcal{B}2 )$,因此$\bigcup\limits_{i=1}^{\infty}A_i \in (\mathcal{B}1 \cap \mathcal{B}2 )$,即满足性质c。\\
  即证得证明两个$\sigma$代数的交仍是$\sigma$代数。
 \end{enumerate}
   
%%---------------------------------------------------------------------------------------------------
\section{【1.13】若$P(A)=\frac{1}{3}$且$P(B^{C})=\frac{1}{4}$, 则$A$, $B$是否不交?试加以解释。}
%===================================================================================================
解:如果$A$, $B$是不交,$P(A \cap B) =0$。 由$P(B^{C})=\frac{1}{4}$得,$P(B)=\frac{3}{4}$,根据\textbf{Bonferroni不等式}, $P(A \cap B) \geqslant P(A)+P(B) - 1 =  \frac{1}{3}  +  \frac{3}{4}  - 1 = \frac{1}{12} > 0$, 因此不交。(这样证明可以吗?)
 
 %%---------------------------------------------------------------------------------------------------
\section{【1.21】橱内有$n$双鞋子,随机抽出2$r$只鞋子(2$r<n$),这些鞋子全都无法配对的概率是多少?}
%===================================================================================================
解: 随机从$2n$只鞋子中抽出2$r$只鞋子的组合数为:${2n \choose 2r}$。鞋子分左鞋或右鞋,如果左右鞋不配也无法进行配对,而左右鞋被选中的概率为$2^{2r}$, 最后要求全都无法配对,必定是在n种不同的鞋子中选$2r$只鞋子,选的组合数为:${n \choose 2r}$,只有这样选才一定能保证完全无法配对,而这$2r$只鞋子可以是左鞋也可以是右鞋,因此这些鞋子全都无法配对的概率是$\frac{{n \choose 2r} 2^{2r}}{{2n \choose 2r}}$
%%---------------------------------------------------------------------------------------------------
\section{【1.22】}
%===================================================================================================
\begin{enumerate}
  \item[(a)]  对全年366天(包含2月29日)抽签,抽出的前180天刚好平均分布于全年12个月的概率是多少?\\
  解:首先对366天中抽出180天出来的概率为${366 \choose 180}$,其次要求180天平均分布在12个月中的话,意味着每个月中都均匀的抽了15天,设每个月的天数集合为$N_i=\{n_1,n_2,...,n_{12}\}$,其实是已知的,$N_i=\{31,29,31,...,31\}$, 抽样组合数为:$\prod\limits_{i=1}^{12}{n_i \choose 15}$,其中$N_i=\{31,29,31,...,31\}$。因此抽出的前180天刚好平均分布于全年12个月的概率是$\frac{\prod\limits_{i=1}^{12}{n_i \choose 15}}{{366 \choose 180}}$,其中$N_i=\{31,29,31,...,31\}$。
  
  \item[(b)]  抽出的前30天都不在九月份的概率是多少?
  解:9月是30天,现在进行30次抽样,组合数为:${366 \choose 30}$,要求不抽到9月的任意一天,那么第一次抽样组合数为:$\frac{366-30}{366}$, 第二次抽样组合数为:$\frac{366-30-1}{366}$,依次类推,第30次抽样的组合数为:$\frac{366-30-29}{366}$。总的概率为$\frac{\prod\limits_{i=1}^{30}{\frac{366-30-i+1}{366}}}{{366 \choose 30}}$。
  
 \end{enumerate}  

%%---------------------------------------------------------------------------------------------------
\section{【1.29】}
%===================================================================================================
\begin{enumerate}
  \item[(a)]  对于例1.2.20,枚举构成无序样本$\{4,4,12,12\}$和$\{2,9,9,12\}$的有序样本。\\
  解:枚举结果如下表:
  \begin{table}[htbp]
\caption{枚举}
\begin{center}
\begin{tabular}{|c|c|}
\hline %绘制一条水平的线
无序样本  &  有序样本 \\
\hline
$\{4,4,12,12\} $ & (4,4,12,12), (4,12,12,4), (4,12,4,12), (12,4,12,4), (12,4,4,12), (12,12,4,4)\\
\hline
\multirow{2}{*}{$\{2,9,9,12\}$} & (2,9,9,12), (2,9,12,9), (2,12,9,9), (9,2,9,12),(9,2,12,9), (9,9,2,12)\\
 & {(9,9,12,2),(9,12,2,9),(9,12,9,2), (12,2,9,9), (12,9,2,9), (12,9,9,2)}\\
\hline
\end{tabular}
\end{center}
\label{default}
\end{table}%

  \item[(b)]  假设我们有六个数$\{1,2,7,8,14,20\}$,问有放回的抽取时,抽的无序样本$\{2,7,7,8,14,14\}$的概率是多少?\\
  解:无序样本的抽样中,$m$个互不相同的数字占据的位置分别为$k_1,k_2,k_m$, 有序的样本总数为$\frac{k!}{k_1! k_2! ... k_m! }$ ,即为$\frac{6!}{1! 2!  1! 2! }=180$ , 有序样本的总数为$n^r$ 即$6^6$, 即可得抽的无序样本$\{2,7,7,8,14,14\}$的概率是$\frac{180}{6^6}$ 
  \item[(c)]  设对m个数分别有放回的抽取$k_{1}, k_{2}, ... ,k_{m}$次,得到一个大小为$k(k_{1}+k_{2}+...+k_{m}=k)$的无序样本。证明这个样本由$\frac{k!}{k_{1}! k_{1}! ... k_{m}!}$个有序样本构成。\\
  证明: 设抽样次数为$k$, 那么有序抽样的总数为$k!$, 由于有$m$个不同的数字,其内部的排列组合次数分别为:$k_{1}!, k_{2}!, ... ,k_{m}!$,由于同一个数字处在相邻的多个位置只会得到一种有序数,因此有序抽样的总数$k!$中会有$\prod\limits_{i=1}^{m}{k_i!}$个重复,因此有序样本的总数应该为$\frac{k!}{k_{1}! k_{1}! ... k_{m}!}$。
  \item[(d)]  证明多项式的系数的个数(即自助法的样本总数)为${k+m-1 \choose k}$, 即:$\sum\limits_{k_1,k_2,...,k_m}{I_{\{k_{1}+k_{2}+...+k_{m}=k\}}}= {k+m-1 \choose k}$\\
 
 证明: 这个自助法的抽样问题可以简化成一个物体装箱问题,总数$k$可以看做所有的物体,即用$m$个互不相同的数字表示$m$个箱子,即$k$个物体装到$m$个箱子中,那么这种装箱的总数为${k+m-1 \choose k}$。
 
 \end{enumerate}  



%%---------------------------------------------------------------------------------------------------
\section{【1.38】证明下列各命题(假定每个条件事件的概率都是正的)}
%===================================================================================================
\begin{enumerate}
  \item[(a)]  若$P(B)=1$, 则对任意A,有$P(A|B)=P(A)$.\\
  证明:由课本定理1.2.11a可得$P(A)=P(A \cap B) + P(A \cap B^{C})$,\\
  又因为$A \cap B^{C} \subset B^{C}$,且$P(B^{C})=1-P(B)=1-1=0$,\\
  故,$P(A \cap B^{C})=0$ ,所以$P(A)=P(A \cap B) $,\\
  因此$P(A|B)=\frac{P(A \cap B)}{P(B)}=\frac{P(A)}{1}=P(A)$,\\
  故,若$P(B)=1$, 则对任意A,有$P(A|B)=P(A)$得证。
  
  \item[(b)]  若$A \subset B$, 则$P(B|A)=1$,且$P(A|B)=P(A)/P(B)$.\\
  证明:由$A \subset B$可得 $A \cap B = A$, \\
  进而$P(B|A)=\frac{P(A \cap B)}{P(A)}=\frac{P(A)}{P(A)}=1$,\\
  同样可得$P(A|B)=\frac{P(A \cap B)}{P(B)}=\frac{P(A)}{P(B)}$
  故,若$A \subset B$, 则$P(B|A)=1$,且$P(A|B)=P(A)/P(B)$得证。
  
  \item[(c)]  若$A, B$不交,则$P(A|A\cup B)=\frac{P(A)}{P(A)+P(B)}$.\\
  证明:若$A, B$不交,则有$P(A \cup B)=P(A)+P(B)$(有限可加性公理),\\
  同时有$A \cap (A \cup B) = A$.\\
  因此,$P(A|A\cup B)=\frac{A \cap (A\cup B)}{P(A\cup B)}=\frac{P(A)}{P(A)+P(B)}$,\\
  故,若$A, B$不交,则$P(A|A\cup B)=\frac{P(A)}{P(A)+P(B)}$得证。
  
  \item[(d)]  $P(A \cap B \cap C)=P(A|B \cap C)P(B|C)P(C)$.\\
  证明:根据结合律,有$P(A \cap B \cap C)=P(A \cap (B \cap C))$,\\
  $\Leftrightarrow$\\
$=P(A| B \cap C)P(B \cap C)$, 条件概率公式\\
  $\Leftrightarrow$\\
  $=P(A| B \cap C)P(B|C)P(C)$, 条件概率公式\\
  故,“$P(A \cap B \cap C)=P(A|B \cap C)P(B|C)P(C)$”得证。
 \end{enumerate}  

%%---------------------------------------------------------------------------------------------------
\section{【1.41】同1.3.6,仍假定电报信号中点信号与化信号的比例是3:4,此外由于线路干扰导致$\frac{1}{4}$的点信号被误传为划信号,$\frac{1}{3}$的划信号被误传为点信号。}
%===================================================================================================
\begin{enumerate}
  \item[(a)]  若收到的是划信号,试问发送的是划信号的概率是多少?\\
  解:$P(发划|收划)=\frac{P( 收划|发划 )P(收划)}{P( 收划|发划 )P(收划)+P(收划|收点)P(收点)}=\frac{(2/3)(4/7)}{(2/3)(4/7)+(1/4)(3/7)}=\frac{32}{41}$
  \item[(b)]  假定信号的连续发送是独立事件,若收到的消息为:点-点,则发送方发送四种可能的消息的概率分布如何?\\
  解:设发送的两种消息类型为$A,B$,因为假设是独立事件,因此有,$P(A-B|点-点)=P(A|点)P(B|点)$, \\
  因此四种可能的消息概率:$P(发点-发点|收点-收点)=P(发点|收点)^2$,$P(发划-发点|收点-收点)=P(发划|发点)P(收点|收点)$, $P(发点-发划|收点-收点)=P(发点|收点)P(发划|收点)$,$P(发划-发划|收点-收点)=P(发划|收点)^2$。\\
  根据题意可知,$P(收点|发点)=1-1/4=\frac{3}{4}$,\\  现在求$P(发点|收点)$和$P(发划|收点)$,\\
  $P(发点|收点)\\=\frac{P(收点|发点)P(发点)}{P(收点)}=\frac{P(收点|发点)P(发点)}{P(收点|发点)P(发点)+P(收点|发划)P(发划)}=\frac{(3/4)(3/7)}{(3/4)(3/7)+(1/3)(4/7)}=\frac{27}{43}$,\\
  $P(发划|收点)\\=\frac{P(收点|发划)P(发划)}{P(收点)}=\frac{P(收点|发划)P(发划)}{P(收点|发点)P(发点)+P(收点|发划)P(发划)}=\frac{(1/3)(4/7)}{(3/4)(3/7)+(1/3)(4/7)}=\frac{16}{43}$,\\
  因此,\\
  $P(发点-发点|收点-收点)=P(发点|收点)^2=(\frac{27}{43})^2$,\\
  $P(发划-发点|收点-收点)=P(发划|收点)P(发点|收点)=\frac{16}{43}\frac{27}{43}$,\\
   $P(发点-发划|收点-收点)=P(发点|收点)P(发划|收点)=\frac{27}{43}\frac{16}{43}$,\\
   $P(发划-发划|收点-收点)=P(划|收点)^2=(\frac{16}{43})^2$
  
 \end{enumerate}  

%%---------------------------------------------------------------------------------------------------
\section{【1.49】称累积分布函数$F_{X}$ \textbf{随机大于}(stochastically greater)累积分布函数$F_{Y}$,如果对任意$t$都有$F_{X}(t) \leqslant F_{Y}(t)$。}
证明:由题意可知,对于任意的$t$,都有$F_{X}(t) \le F_{Y}(t)$,\\
所以有 $1-F_{X}(t) \ge 1-F_{Y}(t)$,\\
因此,$P(X>t)=1-P(X \le t)=1-F_{X}(t) $,\\
$ \ge 1-F_{Y}(t) = 1-P(Y \le t)=P(Y>t)$.\\
那么对于存在的某个$t'$,有$P(X>t‘)=1-P(X \le t’)=1-F_{X}(t‘) $,\\
$ > 1-F_{Y}(t’) = 1-P(Y \le t‘)=P(Y>t)$.\\

%==================================================================================================

%%---------------------------------------------------------------------------------------------------
\section{【1.50】验证式(1.5.4)}
解:式1.5.4为$\sumn 1 t^{k-1}=\frac{1-t^n}{1-t}$, \\
假设为n的时候成立,有$\sumn k t^{k-1}=\frac{1-t^n}{1-t}$,\\
那么对于$n+1$, 有$\sum\limits_{k=1}^{n+1} t^{k-1}=\sumn k t^{k-1} + t^n=\frac{1-t^{n+1}}{1-t}$, \\
由此能够推导出该式对于所有的$n \in N^*$都成立。 
%===================================================================================================


%%---------------------------------------------------------------------------------------------------
\section{【1.52】证明$g(x)$是一个概率密度函数。}
证明: 因为$F(x_0)<1$, 所以$1-F(x_0) >0$ ,\\
又因为$f(x)$是一个概率密度函数,\\
因此对于任意的$x$,都有$f(x)>0$,\\
从而有对于任意的$x \ge x_0$,都有 $\frac{f(x)}{[1-F(x_0) ]} \ge 0$,\\
另外$\int_{x_0}^{\infty}g(x)dx=\int_{x_0}^{\infty}\frac{f(x)}{1-F(x_0)dx}=\frac{1-F(x_0)}{1-F(x_0)}=1$,\\
故$g(x)$是一个概率密度函数得证。
%===================================================================================================

\section{【exercise1-3】A random variable $X$ is said to have a Gamma distribution if its pdf is: \\
     $$f(x|\alpha,\theta)=\frac{1}{\Gamma(\alpha)\theta^\alpha}x^{\alpha-1}e^{-x/\theta},x \in [0,\infty), \alpha >0,\theta>0$$}
      \begin{enumerate}
      \item[(a)] Verify $f(x|\alpha,\theta)$ is a valid pdf.\\
      证明:两个性质不难证明性质a, 即$f_{X}(x) \ge 0$。\\
      不难推导,$$f_{X}(x|\alpha,\theta)=\int_{-\infty}^{\infty}f_{X}(x|\alpha,\theta) dx = \int_{-\infty}^{\infty}\frac{1}{\Gamma(\alpha)\theta^\alpha}x^{\alpha-1}e^{-x/\theta} dx \approx 1$$ 即可证$f(x|\alpha,\theta)$ 是概率密度函数。
      \item[(b)] Find \textbf{mean} and \textbf{variance} of $X$.
      解: 先求期望,$$\mathbb{E}(x)=\int_{0}^{\infty}{\frac{1}{\Gamma(\alpha)\theta^\alpha}x^{\alpha-1}e^{-x/\theta}}dx = \frac{\Gamma(\alpha+1)}{\Gamma(\alpha)} = \alpha$$
      	再求平方的期望,$$\mathbb{E}(x^2)=\int_{0}^{\infty}{\frac{1}{\Gamma(\alpha)\theta^\alpha}x^{2\alpha-2}e^{-2x/\theta}}dx = \frac{\Gamma(\alpha+2)}{\Gamma(\alpha)} = \alpha(\alpha+1)$$,
	因此方差$variance(x)=\mathbb{E}(x^2)-(\mathbb{E}(x))^2 =\alpha$
      \item[(c)] Let $Y=1/X$. What is the pdf of $Y$?($Y$ is said to have an inverse gamma distribution)
      解:令$Y=1/X$, 则有,$\frac{dx}{dy}=\frac{1}{y^2}$, 且$Y$的概率密度函数为:\\
      $$f_{Y}(y|\alpha,\theta)=f_{X}(y|\alpha,\theta) \frac{dx}{dy}$$
      $$f_{Y}(y|\alpha,\theta)=\frac{1}{\Gamma(\alpha)\theta^\alpha}(\frac{1}{y})^{\alpha-1}e^{-1/(\theta y)} \mid  \frac{1}{y^2}\mid$$
      $$f_{Y}(y|\alpha,\theta)=\frac{1}{\Gamma(\alpha)\theta^\alpha}y^{-\alpha-1}e^{-1/(\theta y)} $$
      用$\beta$替换$\theta^{-1}$得:
       $$f_{Y}(y|\alpha,\beta)=\frac{\beta^{\alpha}}{\Gamma(\alpha)}y^{-\alpha-1}e^{-\beta/y} $$

      因此,$Y$的概率密度函数为逆Gamma分布。
      \end{enumerate}

 %%---------------------------------------------------------------------------------------------------


%%%%%%%%%%%%%%%%%%%%%%%%%%%%%%%%%%%%%%%%%%%%%%%%%%%%%%%%%%%%%%%%%%%%%%%%%%%%%%%%%%%%%%%%%%%%%%
  \end{CJK*}
\end{document}
